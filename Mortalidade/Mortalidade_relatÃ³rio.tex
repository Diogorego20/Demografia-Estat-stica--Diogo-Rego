
\documentclass[12pt]{article}
\usepackage[brazil]{babel}
\usepackage[utf8]{inputenc}
\usepackage{graphicx}
\usepackage{geometry}
\usepackage{titlesec}
\usepackage{indentfirst}
\usepackage{setspace}
\usepackage{lmodern}
\usepackage{caption}
\usepackage{fancyhdr}
\usepackage{hyperref}
\usepackage{float}

\geometry{a4paper, margin=2.5cm}
\titleformat{\section}{\normalfont\Large\bfseries}{\thesection}{1em}{}
\titleformat{\subsection}{\normalfont\large\bfseries}{\thesubsection}{1em}{}
\setlength{\parindent}{1.5em}
\setlength{\parskip}{0.5em}
\onehalfspacing
\pagestyle{fancy}
\fancyhf{}
\rhead{Discente: Diogo Rego - Matrícula: 20240045381}
\lhead{Demografia I - UFPB}
\rfoot{\thepage}

\begin{document}

% Capa
\begin{titlepage}
    \centering
    {\Large UNIVERSIDADE FEDERAL DA PARAÍBA\\}
    {\large CENTRO DE CIÊNCIAS EXATAS E DA NATUREZA\\}
    {\large DEPARTAMENTO DE ESTATÍSTICA\\}
    {\large DISCIPLINA: DEMOGRAFIA I\\[2cm]}
    {\Huge \textbf{Análise Comparativa dos Indicadores de Mortalidade no Brasil: 2010 e 2022}\\[2cm]}
    {\large Docente: Everlane Suane de Araújo da Silva\\}
    {\large Discente: Diogo da S. Rego\\}
    {\large Matrícula: 20240045381\\[2cm]}
    {\large João Pessoa – PB\\}
    {\large Maio, 2025\\}
\end{titlepage}

% Resumo
\section*{Resumo}
Este relatório apresenta uma análise comparativa dos principais indicadores de mortalidade no Brasil nos anos de 2010 e 2022, com destaque para as diferenças por sexo. Utilizando dados oficiais do IBGE e DATASUS, foram gerados gráficos explicativos para cada indicador, incluindo taxa bruta de mortalidade, taxa específica, taxa padronizada, mortalidade infantil, neonatal, pós-neonatal, perinatal, materna e por causas específicas. A metodologia adotada envolveu a coleta de dados secundários e a aplicação de técnicas estatísticas descritivas. Os resultados evidenciam mudanças significativas nos padrões de mortalidade ao longo do período analisado. Este estudo contribui para a compreensão da dinâmica demográfica brasileira e subsidia políticas públicas voltadas à saúde da população.

\textbf{Palavras-chave:} Mortalidade, Indicadores Demográficos, Brasil.

% Introdução
\section{Introdução}
A mortalidade é um dos principais componentes da dinâmica demográfica e reflete diretamente as condições de saúde e desenvolvimento de uma população. Este relatório tem como objetivo analisar comparativamente os indicadores de mortalidade no Brasil nos anos de 2010 e 2022, considerando as diferenças por sexo. A metodologia adotada baseia-se na análise estatística de dados secundários provenientes de fontes oficiais. O trabalho está estruturado em seções que abordam a metodologia, os resultados obtidos com gráficos explicativos e a discussão dos achados, seguidos da conclusão e referências bibliográficas.

% Metodologia
\section{Procedimento Metodológico}
A metodologia utilizada neste estudo consiste na coleta de dados de mortalidade dos anos de 2010 e 2022, segmentados por sexo, a partir das bases do IBGE e DATASUS. Foram calculados os seguintes indicadores: taxa bruta de mortalidade, taxa específica, taxa padronizada, mortalidade infantil, neonatal, pós-neonatal, perinatal, materna e por causas específicas. Os dados foram tratados e visualizados por meio de gráficos gerados com ferramentas estatísticas, permitindo uma análise comparativa entre os dois períodos.

% Resultados e Discussão
\section{Resultados e Discussão}
A seguir são apresentados os gráficos gerados para cada indicador de mortalidade, acompanhados de suas respectivas legendas explicativas.

\begin{figure}[H]
    \centering
    \begin{minipage}[b]{0.48\textwidth}
        \includegraphics[width=\textwidth]{generated_image-4.png}
        \caption{Taxa Bruta de Mortalidade (TBM) por sexo nos anos de 2010 e 2022.}
    \end{minipage}
    \hfill
    \begin{minipage}[b]{0.48\textwidth}
        \includegraphics[width=\textwidth]{generated_image-5.png}
        \caption{Taxa Específica de Mortalidade (TEM) por sexo nos anos de 2010 e 2022.}
    \end{minipage}
\end{figure}

\begin{figure}[H]
    \centering
    \begin{minipage}[b]{0.48\textwidth}
        \includegraphics[width=\textwidth]{generated_image-6.png}
        \caption{Taxa Bruta de Mortalidade Padronizada pelo Processo Direto.}
    \end{minipage}
    \hfill
    \begin{minipage}[b]{0.48\textwidth}
        \includegraphics[width=\textwidth]{generated_image-7.png}
        \caption{Taxa de Mortalidade Infantil (TMI) por sexo nos anos de 2010 e 2022.}
    \end{minipage}
\end{figure}

\begin{figure}[H]
    \centering
    \begin{minipage}[b]{0.48\textwidth}
        \includegraphics[width=\textwidth]{generated_image-8.png}
        \caption{Taxa de Mortalidade Neonatal (TMN) por sexo nos anos de 2010 e 2022.}
    \end{minipage}
    \hfill
    \begin{minipage}[b]{0.48\textwidth}
        \includegraphics[width=\textwidth]{generated_image-9.png}
        \caption{Taxa de Mortalidade Pós-Neonatal (TMPN) por sexo nos anos de 2010 e 2022.}
    \end{minipage}
\end{figure}

\begin{figure}[H]
    \centering
    \begin{minipage}[b]{0.48\textwidth}
        \includegraphics[width=\textwidth]{generated_image-10.png}
        \caption{Taxa de Mortalidade Perinatal (TMP) por sexo nos anos de 2010 e 2022.}
    \end{minipage}
    \hfill
    \begin{minipage}[b]{0.48\textwidth}
        \includegraphics[width=\textwidth]{generated_image-11.png}
        \caption{Taxa de Mortalidade Materna (TMM) por sexo nos anos de 2010 e 2022.}
    \end{minipage}
\end{figure}

\begin{figure}[H]
    \centering
    \begin{minipage}{0.48\textwidth}
        \includegraphics[width=\textwidth]{generated_image-12.png}
        \caption{Taxa de Mortalidade por Doenças do Aparelho Circulatório.}
    \end{minipage}
    \hfill
    \begin{minipage}{0.48\textwidth}
        \includegraphics[width=\textwidth]{}
\begin{figure}
            \centering
            \includegraphics[width=0.9\linewidth]{generated_image-12.png}
            \label{fig:placeholder}
        \end{figure}
                \caption{Taxa de Mortalidade por Neoplasias.}
    \end{minipage}
\end{figure}

% Conclusão
\section{Conclusão}
A análise dos indicadores de mortalidade entre os anos de 2010 e 2022 revela mudanças significativas nos padrões de óbitos no Brasil, com destaque para a redução da mortalidade infantil e materna. As diferenças por sexo também evidenciam desigualdades que devem ser consideradas na formulação de políticas públicas. O estudo alcançou seus objetivos ao apresentar uma visão comparativa e fundamentada dos principais indicadores demográficos relacionados à mortalidade.

% Referências
\section{Referências}
\begin{itemize}
    \item BRASIL. Ministério da Saúde. DATASUS – Departamento de Informática do SUS. Disponível em: \url{http://www.datasus.gov.br}. Acesso em: maio 2025.
    \item INSTITUTO BRASILEIRO DE GEOGRAFIA E ESTATÍSTICA – IBGE. Indicadores de Mortalidade. Disponível em: \url{https://www.ibge.gov.br}. Acesso em: maio 2025.
    \item ALVES, J. E. D. A dinâmica demográfica brasileira: uma análise dos principais indicadores. Revista Brasileira de Estudos Populacionais, v. 29, n. 1, p. 5-20, 2012.
    \item CARVALHO, J. A. M.; GARCIA, R. A. Demografia: uma abordagem introdutória. Belo Horizonte: UFMG, 2003.
\end{itemize}

\end{document}
